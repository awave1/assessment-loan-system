% !TEX program = xelatex

\documentclass{article}
\usepackage[utf8]{inputenc}
\usepackage{fontspec}
\setmainfont{SFNS Display}

\usepackage{geometry}
\usepackage{amsthm}
\usepackage{amsmath}
\usepackage{setspace}
\onehalfspacing
\usepackage[
  colorlinks=true,
  linkcolor=blue,
  filecolor=blue,
  urlcolor=blue,
]{hyperref}

\usepackage{color}
\definecolor{pblue}{rgb}{0.13,0.13,1}
\definecolor{pgreen}{rgb}{0,0.5,0}
\definecolor{pred}{rgb}{0.9,0,0}
\definecolor{pgrey}{rgb}{0.46,0.45,0.48}
\definecolor{bg}{rgb}{0.96, 0.96, 0.96}

\usepackage{listings}
\lstset{language=Java,
  showspaces=false,
  showtabs=false,
  breaklines=true,
  showstringspaces=false,
  breakatwhitespace=true,
  stepnumber=1,
  numbers=left,
  commentstyle=\color{pgreen},
  keywordstyle=\color{pblue},
  stringstyle=\color{pred},
  basicstyle=\ttfamily,
  moredelim=[il][\textcolor{pgrey}]{\$\$},
  moredelim=[is][\textcolor{pgrey}]{\%\%}{\%\%},
  backgroundcolor = \color{bg}
}

\renewcommand{\b}[1]{\textbf{#1}}
\renewcommand{\i}[1]{\textit{#1}}
\newcommand{\code}[1]{\texttt{#1}}
% \newcommand{\gh_commit}[1]{\href{https://github.com/awave1/assessment-loan-system/commit/#1}{#1}}

% \newcommand*{\gh}[2][]{\href{https://github.com/awave1/assessment-loan-system/commit/#2}%
%                               {\ifthenelse{\equal{#1}{}}{#2}{#1}}}

\newcommand{\gh}[1]{%
  \href{https://github.com/awave1/assessment-loan-system/commit/#1}{#1}%
}

\title{\b{CPSC 501 -- Assignment 1}}

\author{\b{Artem Golovin} \\ 30018900}
\date{}

\begin{document}
\maketitle

% 1.What code in which files was altered. (don’t include the full class files, only the parts relevant to the refactoring).
% 2.What needed to be improved? That is, what “bad code smell” was detected? Use the terminology found in Chapter 3 of the Fowler text available from the course website/D2L.
% 3.What refactoring was applied? What steps did you follow? Use the terminology and mechanics outlined in the Fowler text and illustrate the process with well-chosen code fragments taken from particular versions of your system.
% 4.What code in what files was the result of the refactoring.
% 5.How was the code tested?
% 6.Why is the code better structured after the refactoring? Does the result of the refactoring suggest or enable further refactorings?

\section*{Overview}

The project that was refactored for this assignment is \href{https://github.com/iKeirNez/assessment-loan-system}{iKeirNez/assessment-loan-system}. To preform refactoring on the project, the following fork was created \href{https://github.com/awave1/assessment-loan-system}{awave1/assessment-loan-system}. The fork contains two branches: \code{master} and \code{refactor}. The initial work is left untouched in \code{master} branch and the refactoring was done in \code{refactor} branch. To simplify the access, \code{refactor} branch is default (main) branch.

Original project did not include any README or instructions how to get up and running with the project, therefore I included \href{https://github.com/awave1/assessment-loan-system/blob/refactor/README.md}{README}, describing the steps necessary to build, run, and test the code. To make it easier to manage dependencies and to build the project, I added support for \href{https://gradle.org/}{gradle}.

  \subsection*{Refactoring structure}

  The refactorings that were made in the project, can be categorized into three categories: \b{minor}, \b{medium} and \b{major}. \b{Minor} refactorings were usually made together with \b{major} and \b{minor} refactorings, thus no commits were directly dedicated to \b{minor} refactorings.

\newpage

\section*{Major changes}

This section explains and displays the number of \b{major} changes that were made to the project during refactoring process. Each major change is associated with commit hash.

\subsection*{Using Visitor pattern to eliminate \code{instanceof} checking (\gh{2534ae1e})}
% https://github.com/awave1/assessment-loan-system/commit/2534ae1e8b711ad62fca8e783b861151951a8f02
% 1.What code in which files was altered. (don’t include the full class files, only the parts relevant to the refactoring).
This substantial refactoring was done at the end when I noticed a block of \code{instanceof} checks that were doing the same thing. To eliminate such code smell and make the code more robust, we make use Visitor pattern. That is, we'll create a special Visitor class that will ``visit'' all required methods, thus removing \code{instanceof} checking and abstracting the functionality. To implement this pattern, I added an abstract method to abstract \code{Item} class:

\begin{lstlisting}[language=Java]
public abstract class Item {
  // ...
  public abstract SelectItem<? extends Item> accept(MenuVisitor visitor);
}
\end{lstlisting}

\noindent As you can see, there's a parameter that will need to be passed, \code{MenuVisitor visitor}. \code{MenuVisitor} is our visitor that will be calling specific functions, declared in the \code{MenuVisitor} class. For example:

\begin{lstlisting}[language=Java]
// MenuVisitor.java
public class MenuVisitor implements Visitor {
  // ...
  @Override
  public SelectItem<Book> addBook(Book book) {
    return new SelectItem<>(book, BOOK_MANAGER, BOOK_MANIPULATOR);
  }
}

// Book.java
public class Book extends Item {
  // ...
  @Override
  public SelectItem<Book> addBook(Book book) {
    return new SelectItem<>(book, BOOK_MANAGER, BOOK_MANIPULATOR);
  }
}
\end{lstlisting}

\noindent Therefore, now instead of performing explicit \code{instanceof} checks, we can do this:

\begin{lstlisting}[language=Java]
  MenuVisitor menuVisitor = new MenuVisitor();
  menuVisitor.build(menu, stockRepo.getAll());
\end{lstlisting}

\noindent and \code{MenuVisitor} will take care of which object to create.

% 2.What needed to be improved? That is, what “bad code smell” was detected? Use the terminology found in Chapter 3 of the Fowler text available from the course website/D2L.
As was stated above, this improves the code by removing excessive \code{instanceof} checking with much cleaner pattern that could be used in other potential places where a lot of \code{instanceof}'s is used. This improves the readability of the code by encapsulating the details of implementation.

% 3.What refactoring was applied? What steps did you follow? Use the terminology and mechanics outlined in the Fowler text and illustrate the process with well-chosen code fragments taken from particular versions of your system.
To perform this refactor, \code{MenuVisitor} and \code{Visitor} were introduced. \code{MenuVisitor} class implements the \code{Visitor} interface that contains ``visit'' methods. Therefore, we introduced a new class and a new interface to Replace Conditional with Polymorphism.

% 4.What code in what files was the result of the refactoring.

The following files were changed and added as a result of this refactoring:
\begin{itemize}
  \item \href{https://github.com/awave1/assessment-loan-system/commit/2534ae1e8b711ad62fca8e783b861151951a8f02#diff-54a69d7c35e29e31112b9482279f7728}{src/main/java/com/keirnellyer/glencaldy/item/Book.java}
  \item \href{https://github.com/awave1/assessment-loan-system/commit/2534ae1e8b711ad62fca8e783b861151951a8f02#diff-ffa15ba834ac42bda93108046f2e3e1d}{src/main/java/com/keirnellyer/glencaldy/item/Disc.java}
  \item \href{https://github.com/awave1/assessment-loan-system/commit/2534ae1e8b711ad62fca8e783b861151951a8f02#diff-1fa65bd290d78e7f9e8a0983709d279f}{src/main/java/com/keirnellyer/glencaldy/item/Item.java}
  \item \href{https://github.com/awave1/assessment-loan-system/commit/2534ae1e8b711ad62fca8e783b861151951a8f02#diff-fd27721b429fa9fc4dbaf6aabcd2512d}{src/main/java/com/keirnellyer/glencaldy/item/Journal.java}
  \item \href{https://github.com/awave1/assessment-loan-system/commit/2534ae1e8b711ad62fca8e783b861151951a8f02#diff-3d39eb9fe1d2e7bb2073c8b6e612eaa2}{src/main/java/com/keirnellyer/glencaldy/item/Video.java}
  \item \href{src/main/java/com/keirnellyer/glencaldy/menu/option/stock/EditStockOption.java}{src/main/java/com/keirnellyer/glencaldy/menu/option/stock/EditStockOption.java}
  \item \href{src/main/java/com/keirnellyer/glencaldy/menu/option/stock/SelectItem.java}{src/main/java/com/keirnellyer/glencaldy/menu/option/stock/SelectItem.java}
  \item \href{https://github.com/awave1/assessment-loan-system/commit/2534ae1e8b711ad62fca8e783b861151951a8f02#diff-73e174fc5f923546624d5c4206572c47}{src/main/java/com/keirnellyer/glencaldy/util/MenuVisitor.java}
  \item \href{https://github.com/awave1/assessment-loan-system/commit/2534ae1e8b711ad62fca8e783b861151951a8f02#diff-ce10c0cc207b42495013fef20915f5ab}{src/main/java/com/keirnellyer/glencaldy/util/Visitor.java}
\end{itemize}

% 5.How was the code tested?
To test this code, I added test \code{Item}s that allowed to build the menu with these items and to test \code{visit} methods. For example,
\begin{lstlisting}[language=Java]
Menu menu = new Menu("Test Menu");
menuVisitor.build(menu, stockRepository.getAll());

assertFalse(menu.getItems().isEmpty());

for (int i = 0; i < stockRepository.getAll().size(); i++) {
  assertNotNull(menu.getItems().get(Integer.toString(i)));
}
\end{lstlisting}

\noindent The rest of the methods were simply tested to make sure that they do not return \code{null}, since \code{Menu} class relies on this \code{MenuVisitor} class.

% 6.Why is the code better structured after the refactoring? Does the result of the refactoring suggest or enable further refactorings?
With this refactoring applied, code smell has been removed and the code became more OO. We no longer do \code{instanceof} checks, which is considered bad design. As well as in the future, this codebase can now use Visitor pattern thus getting rid of ``100 line'' if statements.

\subsection*{Abstracting busy waiting for user input (\gh{bd538f0})}
% 1.What code in which files was altered. (don’t include the full class files, only the parts relevant to the refactoring).
The application relies on input from keyboard. In classes, where the user input is required, the original code included use of \code{do \{ ... \} while(...);} loops, to wait for valid user input. User input is required in \code{Property<T>} (and its subclasses), \code{Menu} and \code{Controller}. As a result, \code{do \{ ... \} while(...);} loop has been moved to separate class called \code{ConsoleInput<T>}, in method \code{public Optional<T> waitForInput(InputWait<T> inputWait)}. \code{InputWait<T>} is an interface with only one method, passing it as a parameter allows us to pass anonymous lambda function (e.g. \code{arg -> \{/* function body */\}}) as a parameter, instead of implementing a method, that was declared in the interface. The following is the implementation of \code{waitForInput} method:

\begin{lstlisting}[language=Java]
public Optional<T> waitForInput(InputWait<T> inputWait) {
  Optional<T> fetchedObj;
  do {
    fetchedObj = inputWait.getInput(this.scanner);
  } while (!fetchedObj.isPresent());

  return fetchedObj;
}
\end{lstlisting}

When \code{waitForInput} is called, we have to supply instance of \code{InputWait}, for example:

\begin{lstlisting}[language=Java]
// Set the scanner
setScanner(scanner);
Optional<Option> option = waitForInput(scnr -> {
  /*
   * Do all the necessary things here, using Scanner scnr variable
   */

  // Return Optional object result
  return Optional.of(obj);
});

//... Optional<Option> option can later be safely unwrapped and used
\end{lstlisting}

% 2.What needed to be improved? That is, what “bad code smell” was detected? Use the terminology found in Chapter 3 of the Fowler text available from the course website/D2L.
By abstracting the \code{do\{...\} while()} loop into a separate method, we got rid of duplicated code and made it more readable. Also other simple classes that need to wait for user input can now inherit this class and call \code{waitForInput}.

% 3.What refactoring was applied? What steps did you follow? Use the terminology and mechanics outlined in the Fowler text and illustrate the process with well-chosen code fragments taken from particular versions of your system.
To do this refactoring, \b{Replace Method with Method object} technique was used. For example, with this technique applied, we got rid of \code{processLogin} method in \code{User} class, so the following code:

\begin{lstlisting}[language=Java]
private User processLogin() {
  User user;
  do {
    System.out.println("Please enter your username.");
    String username = scanner.next();

    System.out.println("Please enter your password.");
    String password = scanner.next();

    user = model.getUserRepository().getExact(username, password);

    if (user == null) { // invalid credentials
      System.out.println("Invalid credentials, please try again.");
    }
  } while (user == null);

  return user;
}
\end{lstlisting}

\noindent was replaced with:
\begin{lstlisting}[language=Java]
Optional<User> user = waitForInput(s -> {
  User usr;
  System.out.println("Please enter your username.");
  String username = scanner.next();

  if (user != null) {
    System.out.println("Please enter your password.");
    String password = scanner.next();
    usr = model.getUserRepository().getExact(username, password);
    if (usr == null) { // invalid credentials
      System.out.println("Invalid credentials, please try again.");
    }

    return Optional.of(usr);
});
\end{lstlisting}

% 4.What code in what files was the result of the refactoring.
The following files were changed and added as a result of this refactoring:
\begin{itemize}
  \item \href{https://github.com/awave1/assessment-loan-system/commit/bd538f05a6fbff8283b397c9a265040ab082542d#diff-1c55a45f266eb74be0a94a96e52a6c2a}{src/main/java/com/keirnellyer/glencaldy/manipulation/property/type/Property.java}
  \item \href{https://github.com/awave1/assessment-loan-system/commit/bd538f05a6fbff8283b397c9a265040ab082542d#diff-d9f6cc6de5b899354d46d7795b29022c}{src/main/java/com/keirnellyer/glencaldy/menu/Menu.java}
  \item \href{https://github.com/awave1/assessment-loan-system/commit/bd538f05a6fbff8283b397c9a265040ab082542d#diff-0f417f930dba04b27baaa509b7b80bbf}{src/main/java/com/keirnellyer/glencaldy/runtime/Controller.java}
  \item \href{https://github.com/awave1/assessment-loan-system/commit/bd538f05a6fbff8283b397c9a265040ab082542d#diff-a28d2a28f9e29d7231bd79421f225da1}{src/main/java/com/keirnellyer/glencaldy/runtime/Controller.java}
  \item \href{https://github.com/awave1/assessment-loan-system/commit/bd538f05a6fbff8283b397c9a265040ab082542d#diff-a28d2a28f9e29d7231bd79421f225da1}{src/main/java/com/keirnellyer/glencaldy/util/ConsoleInput.java}
  \item \href{https://github.com/awave1/assessment-loan-system/commit/bd538f05a6fbff8283b397c9a265040ab082542d#diff-c2879e1c4aeb580d558cae1973565454}{src/main/java/com/keirnellyer/glencaldy/util/InputWait.java}
\end{itemize}

% 5.How was the code tested?
To test \code{ConsoleInput<T>}, I had to mock user input, using \code{ByteArrayInputStream}. There are four tests for this class, located in \code{ConsoleInputTest}. Two tests are used to test \code{ConsoleInput<User>} and one more to test and display functionality using built in object \code{ConsoleInput<String>}.

% 6.Why is the code better structured after the refactoring? Does the result of the refactoring suggest or enable further refactorings?
Adding \code{ConsoleInput<T>} allows us to add ability to use busy-waiting on any class, by utilizing Java Generics.

\subsection*{Add \code{UserInfo} class to eliminate long parameter list for \code{User} classes (\gh{fcaddaa8}, \gh{8fb88595})}
% 1.What code in which files was altered. (don’t include the full class files, only the parts relevant to the refactoring).
This refactoring affected all of the \code{User} classes, so \code{User} and its child classes. All of these classes required a lot of parameters when creating a new object. To eliminate long parameter list, I introduced a \code{UserInfo} class. \code{UserInfo} class uses chaining methods to build information about a required user type. It also contains every possible field that any given \code{User} might use. By doing so, we can use it in all of the \code{User} child classes. As a result, \code{User} now has a \code{UserInfo} field and every set method now has to also set the value for \code{UserInfo} field.

% 3.What refactoring was applied? What steps did you follow? Use the terminology and mechanics outlined in the Fowler text and illustrate the process with well-chosen code fragments taken from particular versions of your system.
To perform this refactoring, I applied ``Extract Class'' technique. Here's the snippet of \code{UserInfo} class

\begin{lstlisting}[language=Java]
public class UserInfo {
  private String username;
  private String password;
  private LocalDate birthDate;
  private String phoneNumber;
  private String address;
  private int staffId;
  private String email;
  private String extension;

  // ... setters and getters

  // Example of a setter
  // As you can see, it returns this to enable chaining setters
  public UserInfo setUsername(String username) {
    this.username = username;
    return this;
  }
}
\end{lstlisting}

\noindent This data class allowed to replace long parameter constructors like these:

\begin{lstlisting}[language=Java]
public Administrative(String username, String password, String address, String phoneNumber, LocalDate birthDate, int id, String email, String extension) {
    super(username, password, address, phoneNumber, birthDate, id, email, extension);
}
\end{lstlisting}

\noindent with this:
\begin{lstlisting}[language=Java]
public Administrative(UserInfo info) {
    super(info);
}
\end{lstlisting}

% 4.What code in what files was the result of the refactoring.
The following files were changed as a result of this refactoring:
\begin{itemize}
  \item \href{https://github.com/awave1/assessment-loan-system/commit/fcaddaa808bff70b463e9c852aa1653040a01ab8?diff=split#diff-5a0e9b14541283daee7c9eed50aa74cc}{src/main/java/com/keirnellyer/glencaldy/manipulation/user/StaffProperties.java}
  \item \href{https://github.com/awave1/assessment-loan-system/commit/fcaddaa808bff70b463e9c852aa1653040a01ab8?diff=split#diff-4a7373ea05df51d34c2533824251b057}{src/main/java/com/keirnellyer/glencaldy/user/Administrative.java}
  \item \href{https://github.com/awave1/assessment-loan-system/commit/fcaddaa808bff70b463e9c852aa1653040a01ab8?diff=split#diff-7338aa10a391d939436dc47f59059277}{src/main/java/com/keirnellyer/glencaldy/user/Casual.java}
  \item \href{https://github.com/awave1/assessment-loan-system/commit/fcaddaa808bff70b463e9c852aa1653040a01ab8?diff=split#diff-7338aa10a391d939436dc47f59059277}{src/main/java/com/keirnellyer/glencaldy/user/Member.java}
  \item \href{https://github.com/awave1/assessment-loan-system/commit/fcaddaa808bff70b463e9c852aa1653040a01ab8?diff=split#diff-87bac8dd50b911a677e2d06aa0e2e5a3}{src/main/java/com/keirnellyer/glencaldy/user/Staff.java}
  \item \href{https://github.com/awave1/assessment-loan-system/commit/fcaddaa808bff70b463e9c852aa1653040a01ab8?diff=split#diff-5e515c5505a969c1bd2e93523ab6c124}{src/main/java/com/keirnellyer/glencaldy/user/User.java}
  \item \href{https://github.com/awave1/assessment-loan-system/commit/fcaddaa808bff70b463e9c852aa1653040a01ab8?diff=split#diff-88553ab84e0d52c8cedbc9e311377b99}{src/main/java/com/keirnellyer/glencaldy/user/UserInfo.java}
\end{itemize}

% 5.How was the code tested?
The testing of this code was done in commit \gh{8fb88595}. To test \code{UserInfo}, I created instance of \code{UserInfo} object and passed it to one of child classes of \code{User} and then, using Java \code{equals} I compared the object I created with instance of it in \code{User} subclass:

\begin{lstlisting}[language=Java]
@Test
void shouldCreateAdministrativeObjectUsingUserInfo() {
  UserInfo userInfo = new UserInfo();
  userInfo
    .setUsername("user")
    .setPassword("pass")
    .setAddress("addr")
    .setPhoneNumber("1243")
    .setBirthDate(LocalDate.of(2019, 3, 3))
    .setStaffId(1)
    .setEmail("email@email.com")
    .setExtension("ext");

  assertNotNull(userInfo);

  Administrative admin = new Administrative(userInfo);

  assertNotNull(admin);
  assertEquals(userInfo, admin.getUserInfo());
}
  
\end{lstlisting}

% 6.Why is the code better structured after the refactoring? Does the result of the refactoring suggest or enable further refactorings?
As was stated before, this code gets rid of long constructors. With \code{UserInfo}, it's easy to create and store information about \code{User}s

\section*{Medium changes}

\subsection*{Replace \code{Repository<K, V>} interface with abstract class (\gh{67287c0}); Introduce \code{add} to Repository classes and \code{getExact} to UserRepository class (\gh{aef6c56})}
% 1.What code in which files was altered. (don’t include the full class files, only the parts relevant to the refactoring).
Initial code included \code{Repository<K, V>} as an interface, however, it didn't make sense for it to be an interface, because there was duplicated code in its child classes. Therefore, by abstracting \code{ArrayList<V> getAll()} and \code{public void add(V value)}, we can reuse this abstract class on any other repository classes, if they need to be created in the future. Another chages that were added, include introducing \code{getExact} method in the \code{UserRepository}. This allowed up to remove \code{findUser} from \code{Controller} class, which clearly did not belong there. As well as adding \code{add(V... values)}, which helped to remove multiple calls when trying to add many things to a repository.

% 2.What needed to be improved? That is, what “bad code smell” was detected? Use the terminology found in Chapter 3 of the Fowler text available from the course website/D2L.
The \code{Repository<K, V>} needed to become an abstract class to eliminate code duplication throughout \code{Repository} children.

% 3.What refactoring was applied? What steps did you follow? Use the terminology and mechanics outlined in the Fowler text and illustrate the process with well-chosen code fragments taken from particular versions of your system.
To do the refactorings, I used Pull Up Method technique as well as Move Method. As was stated before, the following methods were removed from \code{StockRepository} and \code{UserRepository} and were moved to parent \code{Repository}:

\begin{lstlisting}[language=Java]
public void add(V value) { 
  repoContents.add(value);
}

public ArrayList<V> getAll() {
  return repoContents;
}
\end{lstlisting}

\noindent where \code{repoContents} is an \code{ArrayList} that contains repository values. An example of Move Method is the following snippet:
\begin{lstlisting}[language=Java]
public User getExact(String username, String password) {
  return this.repoContents
    .stream()
    .filter(item -> item.getUsername().equals(username) && item.getPassword().equals(password))
    .findFirst()
    .orElse(null);
}
\end{lstlisting}

\noindent That method allowed to remove similar method from \code{Controller} class, called \code{findUser}.

% 4.What code in what files was the result of the refactoring.
The following files were affected by the refactoring:
\begin{itemize}
  \item \href{https://github.com/awave1/assessment-loan-system/commit/aef6c56#diff-db219e7646e5c083f6378ffcd27e73b1}{src/main/java/com/keirnellyer/glencaldy/repository/Repository.java}
  \item \href{https://github.com/awave1/assessment-loan-system/commit/aef6c56#diff-d5fd0798d78d7603accc044c57179943}{src/main/java/com/keirnellyer/glencaldy/repository/UserRepository.java}
  \item \href{https://github.com/awave1/assessment-loan-system/commit/aef6c56#diff-0f417f930dba04b27baaa509b7b80bbf}{src/main/java/com/keirnellyer/glencaldy/runtime/Controller.java}
  \item \href{https://github.com/awave1/assessment-loan-system/commit/aef6c56#diff-d00409191136d5295ea968c2d4446f20}{src/main/java/com/keirnellyer/glencaldy/runtime/Model.java}
\end{itemize}

% 5.How was the code tested?
To test this refactoring, I created a \code{UserRepositoryTest}, where I was able to test both \code{add} and \code{getExact} methods.

% 6.Why is the code better structured after the refactoring? Does the result of the refactoring suggest or enable further refactorings?
This refactor brought better structure to the code by removing duplicates and moving methods to where they actually belong.

\subsection*{TODO (\gh{b6b0c98}, \gh{8a898a0})}
% https://github.com/awave1/assessment-loan-system/commit/b6b0c982f35695c9f9bb0eea222788ba911f6f26
% https://github.com/awave1/assessment-loan-system/commit/8a898a0ca9406ffb32fb65c40485973b1f50e97a
% 1.What code in which files was altered. (don’t include the full class files, only the parts relevant to the refactoring).

% 2.What needed to be improved? That is, what “bad code smell” was detected? Use the terminology found in Chapter 3 of the Fowler text available from the course website/D2L.

% 3.What refactoring was applied? What steps did you follow? Use the terminology and mechanics outlined in the Fowler text and illustrate the process with well-chosen code fragments taken from particular versions of your system.

% 4.What code in what files was the result of the refactoring.

% 5.How was the code tested?

% 6.Why is the code better structured after the refactoring? Does the result of the refactoring suggest or enable further refactorings?

\subsection*{TODO (\gh{ad3147})}
% https://github.com/awave1/assessment-loan-system/commit/ad3147c13281bdfa24cbdba143867c95cdd55d1f
% 1.What code in which files was altered. (don’t include the full class files, only the parts relevant to the refactoring).

% 2.What needed to be improved? That is, what “bad code smell” was detected? Use the terminology found in Chapter 3 of the Fowler text available from the course website/D2L.

% 3.What refactoring was applied? What steps did you follow? Use the terminology and mechanics outlined in the Fowler text and illustrate the process with well-chosen code fragments taken from particular versions of your system.

% 4.What code in what files was the result of the refactoring.

% 5.How was the code tested?

% 6.Why is the code better structured after the refactoring? Does the result of the refactoring suggest or enable further refactorings?

\end{document}
